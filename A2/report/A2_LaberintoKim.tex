\documentclass{article}

\usepackage{amsfonts}
\usepackage{amsmath}
\usepackage{svg} %For including svgs
\svgpath{{../assets/}}
\usepackage[margin=1in]{geometry} %Change margins
\usepackage{hyperref} %For hyperlinks in table of contents and other
\usepackage{float} %For using H in figure
\usepackage{subcaption} %For subfigures
\usepackage{booktabs} %For tables
\usepackage{multirow} %For tables

\usepackage[charsperline=120]{jlcode} %For Julia Code Listing https://github.com/wg030/jlcode

\addtolength{\jot}{1em} %https://tex.stackexchange.com/questions/14679/amsmath-align-environment-row-spacing

\title{Optimization Assignment 2\\Comparison between Algorithms}
\date{Winter 2021}
\author{Kim Paolo Laberinto (7771083)}

\begin{document}
    \maketitle
    \newpage

    \tableofcontents
    \newpage


    \section{Methodology}

    In this report several algorithms were analyzed and compared in terms of their performance metrics.
    The metrics used are the following
    \begin{enumerate}
        \item Number of Function Evaluations
        \item Number of Gradient Function Evaluations
        \item Number of Hessian Function Evaluations
        \item Number of Linear System Solves (e.g. inversions and similar)
    \end{enumerate}

    These performance metrics are significant as depending on the objective function to optimize, 
    the various function evaluations can be very computationally expensive.

    Each algorithm was also tested with several initial vectors. These are found in Table \ref{tab:initial_vectors}.

    \begin{table}[H]
        \centering
        \begin{tabular}{@{}ll@{}}
        \toprule
                                  & \textbf{Vector}                     \\ \midrule
        \textbf{Initial Vector 1} & [ 0.00,  0.00,  0.00,  0.00,  0.00] \\
        \textbf{Initial Vector 2} & [ 0.36,  1.18, -1.01, -1.73, -0.90] \\
        \textbf{Initial Vector 3} & [ 1.07,  1.42,  0.32,  1.83,  0.61] \\
        \textbf{Initial Vector 4} & [ 0.26, -1.20,  0.60,  0.59, -1.77] \\
        \textbf{Initial Vector 5} & [-0.16, -0.81, -1.96, -1.55,  1.37] \\ \bottomrule
        \end{tabular}
        \caption{Table of Initial Vectors used for analysing performance}
        \label{tab:initial_vectors}
    \end{table}

    \subsection{Limitations}

    Note that this report is solely for educational purposes only in possible methods of analysis, 
    and not meant to be used to rigorously compare the various algorithms.

    Performance comparisons are highly dependent on the implementation of the algorithm itself.
    The implementations used in this report are not optimized.

    \subsection{Further Areas to Explore}

    There are other aspects to explore for potential future performance analysis.
    These include:

    \begin{enumerate}
        \item Memory Allocation
        \item Time to run
        \item Profiling the code to examine which lines run the longest
    \end{enumerate}

    \section{Algorithm Performances - Loss vs Iterations}

    In this section, the algorithm specific loss vs iterations plots can be seen, to visually compare how many iterations it takes to reach a specific loss value.
    More analysis of the performance metrics can be found in the Comparison Between Algorithms section.
    
    Note that for the Modified Newton's Method with Levenberg-Marquardt Modification, the plots comparing the influence of the mu parameter can also be seen in this section.

    \subsection{Steepest Descent}

    \subsection{Powell's Conjugate Direction}

    \subsection{Conjugate Gradient Techniques}

    \subsection{Hooke Jeeves}

    \subsection{Nelder-Mead Simplex Search}

    \subsection{Original Newton's Method}

    \subsection{Modified Newton's Method with Levenberg-Marquardt Modification}

    \section{Comparison between Algorithms}

    \subsection{Number of Function Evaluations}

    \appendix
    \section{All Plots}

    \begin{figure}[H]
        \centering
        \includesvg[width=0.5\linewidth]{ConjugateGradientFletcherReeves_LossPlot.svg}
        \caption{ConjugateGradientFletcherReeves_LossPlot.svg}
        \label{fig:ConjugateGradientFletcherReeves_LossPlot.svg}
    \end{figure}
    
    \begin{figure}[H]
        \centering
        \includesvg[width=0.5\linewidth]{ConjugateGradientHestenesStiefel_LossPlot.svg}
        \caption{ConjugateGradientHestenesStiefel_LossPlot.svg}
        \label{fig:ConjugateGradientHestenesStiefel_LossPlot.svg}
    \end{figure}
    
    \begin{figure}[H]
        \centering
        \includesvg[width=0.5\linewidth]{ConjugateGradientPolakRibiere_LossPlot.svg}
        \caption{ConjugateGradientPolakRibiere_LossPlot.svg}
        \label{fig:ConjugateGradientPolakRibiere_LossPlot.svg}
    \end{figure}
    
    \begin{figure}[H]
        \centering
        \includesvg[width=0.5\linewidth]{GradientDescentLossPlot.svg}
        \caption{GradientDescentLossPlot.svg}
        \label{fig:GradientDescentLossPlot.svg}
    \end{figure}
    
    \begin{figure}[H]
        \centering
        \includesvg[width=0.5\linewidth]{HookeJeevesLossPlot.svg}
        \caption{HookeJeevesLossPlot.svg}
        \label{fig:HookeJeevesLossPlot.svg}
    \end{figure}
    
    \begin{figure}[H]
        \centering
        \includesvg[width=0.5\linewidth]{NelderMead_LossPlot.svg}
        \caption{NelderMead_LossPlot.svg}
        \label{fig:NelderMead_LossPlot.svg}
    \end{figure}
    
    \begin{figure}[H]
        \centering
        \includesvg[width=0.5\linewidth]{OriginalNewtonsMethod_ConditionNumberHessianPlot.svg}
        \caption{OriginalNewtonsMethod_ConditionNumberHessianPlot.svg}
        \label{fig:OriginalNewtonsMethod_ConditionNumberHessianPlot.svg}
    \end{figure}
    
    \begin{figure}[H]
        \centering
        \includesvg[width=0.5\linewidth]{OriginalNewtonsMethod_LossPlot.svg}
        \caption{OriginalNewtonsMethod_LossPlot.svg}
        \label{fig:OriginalNewtonsMethod_LossPlot.svg}
    \end{figure}
    
    \begin{figure}[H]
        \centering
        \includesvg[width=0.5\linewidth]{PowellConjugateGradient_LossPlot.svg}
        \caption{PowellConjugateGradient_LossPlot.svg}
        \label{fig:PowellConjugateGradient_LossPlot.svg}
    \end{figure}
    
    \begin{figure}[H]
        \centering
        \includesvg[width=0.5\linewidth]{ModifiedNewtonsWithLM_ConditionNumberHessianPlot_1.svg}
        \caption{ModifiedNewtonsWithLM_ConditionNumberHessianPlot_1.svg}
        \label{fig:ModifiedNewtonsWithLM_ConditionNumberHessianPlot_1.svg}
    \end{figure}
    
    \begin{figure}[H]
        \centering
        \includesvg[width=0.5\linewidth]{ModifiedNewtonsWithLM_ConditionNumberHessianPlot_2.svg}
        \caption{ModifiedNewtonsWithLM_ConditionNumberHessianPlot_2.svg}
        \label{fig:ModifiedNewtonsWithLM_ConditionNumberHessianPlot_2.svg}
    \end{figure}
    
    \begin{figure}[H]
        \centering
        \includesvg[width=0.5\linewidth]{ModifiedNewtonsWithLM_ConditionNumberHessianPlot_3.svg}
        \caption{ModifiedNewtonsWithLM_ConditionNumberHessianPlot_3.svg}
        \label{fig:ModifiedNewtonsWithLM_ConditionNumberHessianPlot_3.svg}
    \end{figure}
    
    \begin{figure}[H]
        \centering
        \includesvg[width=0.5\linewidth]{ModifiedNewtonsWithLM_LossPlot_1.svg}
        \caption{ModifiedNewtonsWithLM_LossPlot_1.svg}
        \label{fig:ModifiedNewtonsWithLM_LossPlot_1.svg}
    \end{figure}
    
    \begin{figure}[H]
        \centering
        \includesvg[width=0.5\linewidth]{ModifiedNewtonsWithLM_LossPlot_2.svg}
        \caption{ModifiedNewtonsWithLM_LossPlot_2.svg}
        \label{fig:ModifiedNewtonsWithLM_LossPlot_2.svg}
    \end{figure}
    
    \begin{figure}[H]
        \centering
        \includesvg[width=0.5\linewidth]{ModifiedNewtonsWithLM_LossPlot_3.svg}
        \caption{ModifiedNewtonsWithLM_LossPlot_3.svg}
        \label{fig:ModifiedNewtonsWithLM_LossPlot_3.svg}
    \end{figure}
    
    \begin{figure}[H]
        \centering
        \includesvg[width=0.5\linewidth]{ModifiedNewtonsWithLM_MatrixCompare_1.svg}
        \caption{ModifiedNewtonsWithLM_MatrixCompare_1.svg}
        \label{fig:ModifiedNewtonsWithLM_MatrixCompare_1.svg}
    \end{figure}
    
    \begin{figure}[H]
        \centering
        \includesvg[width=0.5\linewidth]{ModifiedNewtonsWithLM_MatrixCompare_2.svg}
        \caption{ModifiedNewtonsWithLM_MatrixCompare_2.svg}
        \label{fig:ModifiedNewtonsWithLM_MatrixCompare_2.svg}
    \end{figure}
    
    \begin{figure}[H]
        \centering
        \includesvg[width=0.5\linewidth]{ModifiedNewtonsWithLM_MatrixCompare_3.svg}
        \caption{ModifiedNewtonsWithLM_MatrixCompare_3.svg}
        \label{fig:ModifiedNewtonsWithLM_MatrixCompare_3.svg}
    \end{figure}
    

    \section{Source Code}

    %main
    %makeplots
    %objectivefunction
    %generate_random_inits
    %A1Module
    %A2Module

\end{document}